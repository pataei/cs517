% --------------------------------------------------------------
% This is all preamble stuff that you don't have to worry about.
% Head down to where it says "Start here"
% --------------------------------------------------------------
 
\documentclass[12pt]{article}

\usepackage{amsmath}
\usepackage{amsthm}
\usepackage{amsfonts}
\usepackage{hyperref}
\usepackage{comment}

\usepackage{graphicx}
\usepackage{balance}  % for  \balance command ON LAST PAGE  (only there!)
\usepackage{comment}
\usepackage{accents}
\usepackage{lambda,cc}
\usepackage{mathtools} 
\usepackage{array,multirow}
%\usepackage{hhline}
\usepackage{arydshln}
%\usepackage{color}
%\usepackage[usenames, dvipsnames]{xcolor}
\usepackage{xcolor}
%\usepackage{colortbl}
%\usepackage{booktabs}
\usepackage{mathpartir}
\usepackage{hyperref}
\usepackage[normalem]{ulem}
% commuting diagram packages
\usepackage{tikz}
\usepackage{tikz-cd}
\usetikzlibrary{decorations.pathmorphing}
\usepackage{wrapfig}
%\usepackage{blindtext}% for example text here only
\usepackage[inline]{enumitem} %remove enumerate indent
\usepackage{amssymb}%for arrow labels

\usepackage{listings} % for v-sql example code

\lstset{
   breaklines=true,                                     % line wrapping on
   language=SQL,
%   frame=ltrb,
   framesep=2pt,
   basicstyle=\normalsize,
   keywordstyle=\ttfamily\color{black},
   identifierstyle=\ttfamily\color{black}\bfseries,
   commentstyle=\color{Brown},
   stringstyle=\ttfamily,
   showstringspaces=ture
}

% references
\newcommand{\tabref}[1]{\hyperref[tab:#1]{Table~\ref*{tab:#1}}}
\newcommand{\figref}[1]{\hyperref[fig:#1]{Figure~\ref*{fig:#1}}}
\newcommand{\secref}[1]{\hyperref[sec:#1]{Section~\ref*{sec:#1}}}
\newcommand{\defref}[1]{\hyperref[def:#1]{Definition~\ref*{def:#1}}}
\newcommand{\appref}[1]{\hyperref[app:#1]{Appendix~\ref*{app:#1}}}
\newcommand{\chref}[1]{\hyperref[ch:#1]{Chapter~\ref*{ch:#1}}}
\newcommand{\thmref}[1]{\hyperref[thm:#1]{Theorem~\ref*{thm:#1}}}
\newcommand{\lemref}[1]{\hyperref[lem:#1]{Lemma~\ref*{lem:#1}}}
\newcommand{\exref}[1]{\hyperref[ex:#1]{Example~\ref*{eg:#1}}}

%colors
\definecolor{deepcarminepink}{rgb}{0.94, 0.19, 0.22}%all shared
\definecolor{mediumelectricblue}{rgb}{0.01, 0.31, 0.59}%middlename
\definecolor{frenchblue}{rgb}{0.0, 0.45, 0.73}%lastname
\definecolor{green(munsell)}{rgb}{0.0, 0.66, 0.47}%iceland
\definecolor{violet(ryb)}{rgb}{0.53, 0.0, 0.69}%us,invest
\definecolor{navyblue}{rgb}{0.0, 0.0, 0.5}%v-table
\definecolor{persimmon}{rgb}{0.93, 0.35, 0.0}%iran
\definecolor{Plum}{rgb}{0.78, 0.08, 0.52}%us,bank
\definecolor{ruby}{rgb}{0.88, 0.07, 0.37}%us,bank

\definecolor{light-gray}{gray}{0.95}
\newcommand{\code}[1]{\colorbox{light-gray}{\texttt{#1}}}

\newcommand{\resp}[1]{\ifdefined\color
                        {\color{blue}[#1]}%
                      \else
                        {\emph{[#1]}}%
                      \fi}


%TODO
\usepackage{lipsum}                     % Dummytext
\usepackage{xargs}                      % Use more than one optional parameter
                                        % in a new commands
\usepackage{etoolbox}
\usepackage[colorinlistoftodos,prependcaption,textsize=tiny]{todonotes}
\newcommandx{\eric}[2][1=]{\todo[linecolor=red,backgroundcolor=red!25,bordercolor=red,#1]{#2}}
\newcommandx{\arash}[2][1=]{\todo[linecolor=violet(ryb),backgroundcolor=violet(ryb)!25,bordercolor=violet(ryb),#1]{#2}}
%\newcommandx{\arashResp}[2][1=]{\todo[linecolor=blue,backgroundcolor=violet(ryb)!25,bordercolor=violet(ryb),#1]{#2}}
\newcommand{\arashComment}[1]{\TODO {#1}}
\newcommandx{\responded}[1][1=]{\todo[linecolor=green,backgroundcolor=green!25,bordercolor=green,#1]{responded!}}
\newcommandx{\think}[2][1=]{\todo[linecolor=blue,backgroundcolor=blue!25,bordercolor=blue,#1]{#2}}
\newcommandx{\moredet}[2][1=]{\todo[linecolor=green(munsell),backgroundcolor=green(munsell)!25,bordercolor=green(munsell),#1]{#2}}
\newcommandx{\wrrite}[2][1=]{\todo[linecolor=Plum,backgroundcolor=Plum!25,bordercolor=Plum,#1]{#2}}
\newcommandx{\rewrite}[2][1=]{\todo[linecolor=frenchblue,backgroundcolor=frenchblue!25,bordercolor=frenchblue,#1]{#2}}
\newcommandx{\ensure}[2][1=]{\todo[linecolor=ruby,backgroundcolor=ruby!25,bordercolor=ruby,#1]{#2}}
\newcommandx{\dropit}[2][1=]{\todo[linecolor=persimmon,backgroundcolor=persimmon!25,bordercolor=persimmon,#1]{#2}}
\newcommandx{\maybeAdd}[2][1=]{\todo[linecolor=navyblue,backgroundcolor=navyblue!25,bordercolor=navyblue,#1]{#2}}
\newcommandx{\structure}[2][1=]{\todo[linecolor=yellow,backgroundcolor=yellow!25,bordercolor=yellow,#1]{#2}}
\newcommandx{\dfref}[2][1=]{\todo[linecolor=gray,backgroundcolor=gray!25,bordercolor=gray,#1]{#2}}
\newcommand{\soc}{\rewrite {stream of consciousness}}
\newcommand{\badph}{\rewrite {bad pharagraph}}
\newcommand{\badstory}{\rewrite {bad story}}
\newcommand{\badsent}{\rewrite {bad sentence}}
\newcommand{\tbf}{\wrrite {fill later}}
\newcommand{\point}{\textcolor {green(munsell)}}
\newcommand{\chck}{\rewrite{read the ref to make sure your understanding is right!}}
%\newcommandx{\thiswillnotshow}[2][1=]{\todo[disable,#1]{#2}}

%approaches
\newcommand{\nbf}{NBF}
\newcommand{\ubf}{UBF}
\newcommand{\uav}{UAV}

%encoding of variational schema in the database
\newcommand{\vdbpc}{\OB{\mathit{vdb\_pcs}\left(\mathit{element\_id}, \mathit{pres\_cond} \right)}}
\newcommand{\tablespc}{\OB{\mathit{tables\_pc}\left(\mathit{table\_name},\mathit{pres\_cond}\right)}}


%shorthands
\newcommand{\ChcExpTxt}{Choice expression}
\newcommand{\DimTxt}{Choice dimension}
\newcommand{\ConfigTxt}{Configuration}
\newcommand{\vdbTxt}{variational database}
\newcommand{\VdbTxt}{Variational database}
\newcommand{\vdbInstTxt}{variational database instance}
\newcommand{\VdbInstTxt}{Variational database instance}
\newcommand{\vqTxt}{variational query}
\newcommand{\VqTxt}{Variational query}
\newcommand{\vqsTxt}{variational queries}
\newcommand{\VqsTxt}{Variational queries}
\newcommand{\vschTxt}{variational schema}
\newcommand{\VschTxt}{Variational schema}
\newcommand{\tenvTxt}{type}
\newcommand{\TenvTxt}{Type}
\newcommand{\vctxTxt}{variation context}
\newcommand{\VctxTxt}{Variation context}
\newcommand{\vRelSchTxt}{variational relation schema}
\newcommand{\VrelSchTxt}{Variational relation schema}
\newcommand{\vrelTxt}{variational relation}
\newcommand{\VrelTxt}{Variational relation}
\newcommand{\presCondTxt}{presence condition}
\newcommand{\PresCondTxt}{Presence condition}
\newcommand{\fexpTxt}{feature expression}
\newcommand{\FexpTxt}{Feature expression}
\newcommand{\vCondTxt}{variational condition}
\newcommand{\VcondTxt}{Variational condition}
\newcommand{\vAttListTxt}{variational attribute list}
\newcommand{\VattListTxt}{Variational attribute list}
\newcommand{\optAttTxt}{optional attribute}
\newcommand{\OptAttTxt}{Optional attribute}
\newcommand{\VlistTxt}{Variational list}
\newcommand{\VsetTxt}{Variational set}
\newcommand{\AnnotVsetTxt}{Annotated variational set}
\newcommand{\AnnotVlistTxt}{Annotated variational list}
\newcommand{\VobjTxt}{Variational object}
\newcommand{\AttTxt}{Attribute}
\newcommand{\RelSchTxt}{Relation schema}
\newcommand{\SchTxt}{Schema}
\newcommand{\RelTxt}{Relation}
\newcommand{\DbInstTxt}{Database instance}
\newcommand{\DbTxt}{Database}
\newcommand{\CondTxt}{Condition}
\newcommand{\vrelAlgTxt}{variational relational algebra}
\newcommand{\VrelAlgTxt}{variational relational algebra}
\newcommand{\typeSys}{type system}
\newcommand{\varPres}{variation preserving}
\newcommand{\VarPres}{Variation preserving}
\newcommand{\FTxt}{Feature}
\newcommand{\QTxt}{Query}
%\newcommand{\VcondTxt}{Variational condition}
\newcommand{\fctTxt}{true feature set}
\newcommand{\osemTxt}{selection semantics /selection /semantics /configuration semantics/ configuring}

%needs names
\newcommand{\nZero}{\textbf{N0}}
\newcommand{\nOne}{\textbf{N1}}
\newcommand{\nTwo}{\textbf{N2}}
\newcommand{\nThree}{\textbf{N3}}
\newcommand{\basic}{\OB{\mathsf{basic}}}
\newcommand{\educational}{\OB{\mathsf{educational}}}

%motivating ex names
\newcommand{\employee}{\OB{\mathit{emp}}}
\newcommand{\eng}{\OB{\mathit{eng}}}
\newcommand{\edu}{\OB{\mathit{edu}}}
\newcommand{\base}{\OB{\mathit{base}}}
\newcommand{\oncamp}{\OB{\mathit{onCamp}}}
\newcommand{\online}{\OB{\mathit{onLine}}}
\newcommand{\tI}{\OB{T_i}}
\newcommand{\dI}{\OB{D_i}}
\newcommand{\dOne}{\OB{D_1}}
\newcommand{\dFive}{\OB{D_5}}
\newcommand{\sI}{\OB{S_i}}
\newcommand{\sOne}{\OB{S_1}}
\newcommand{\sTwo}{\OB{S_2}}
\newcommand{\sThree}{\OB{S_3}}
\newcommand{\sFour}{\OB{S_4}}
\newcommand{\sFive}{\OB{S_5}}
\newcommand{\sV}{\OB{S_v}}
\newcommand{\vOne}{\OB{V_1}}
\newcommand{\vTwo}{\OB{V_2}}
\newcommand{\vThree}{\OB{V_3}}
\newcommand{\vFour}{\OB{V_4}}
\newcommand{\vFive}{\OB{V_5}}
\newcommand{\tOne}{\OB{T_1}}
\newcommand{\tTwo}{\OB{T_2}}
\newcommand{\tThree}{\OB{T_3}}
\newcommand{\tFour}{\OB{T_4}}
\newcommand{\tFive}{\OB{T_5}}
\newcommand{\fOne}{\OB{f_1}}
\newcommand{\fTwo}{\OB{f_2}}
\newcommand{\engemp}{\OB{\mathit{engineerpersonnel}}}
\newcommand{\othemp}{\OB{\mathit{otherpersonnel}}}
\newcommand{\pers}{\OB{\mathit{personnel}}}
\newcommand{\empacct}{\OB{\mathit{empacct}}}
\newcommand{\empno}{\OB{\mathit{empno}}}
\newcommand{\name}{\OB{\mathit{name}}}
\newcommand{\hiredate}{\OB{\mathit{hiredate}}}
\newcommand{\titleatt}{\OB{\mathit{title}}}
\newcommand{\deptname}{\OB{\mathit{deptname}}}
\newcommand{\salary}{\OB{\mathit{salary}}}
\newcommand{\job}{\OB{\mathit{job}}}
\newcommand{\deptno}{\OB{\mathit{deptno}}}
\newcommand{\dept}{\OB{\mathit{dept}}}
\newcommand{\managerno}{\OB{\mathit{managerno}}}
\newcommand{\empbio}{\OB{\mathit{empbio}}}
\newcommand{\sex}{\OB{\mathit{sex}}}
\newcommand{\birthdate}{\OB{\mathit{birthdate}}}
\newcommand{\fname}{\OB{\mathit{firstname}}}
\newcommand{\lname}{\OB{\mathit{lastname}}}
\newcommand{\course}{\OB{\mathit{course}}}
\newcommand{\ecourse}{\OB{\mathit{ecourse}}}
\newcommand{\student}{\OB{\mathit{student}}}
\newcommand{\teacher}{\OB{\mathit{teacher}}}
\newcommand{\cname}{\OB{\mathit{coursename}}}
\newcommand{\tno}{\OB{\mathit{teacherno}}}
\newcommand{\sno}{\OB{\mathit{studentno}}}
\newcommand{\cno}{\OB{\mathit{courseno}}}
\newcommand{\class}{\OB{\mathit{class}}}
\newcommand{\timeatt}{\OB{\mathit{time}}}
\newcommand{\teach}{\OB{\mathit{teach}}}
\newcommand{\take}{\OB{\mathit{take}}}
\newcommand{\grade}{\OB{\mathit{grade}}}
\newcommand{\isstudent}{\OB{\mathit{std}}}
\newcommand{\isteacher}{\OB{\mathit{instr}}}
\newcommand{\studentnum}{\OB{\mathit{stdnum}}}
\newcommand{\teachernum}{\OB{\mathit{instrnum}}}
\newcommand{\vI}{\OB{V_i}}
% separator
\newcommand{\rSep}{\rn{-}}

\newcommand*{\rE}{\rSep\rn{E}}
\newcommand*{\rRefl}{\rn{Refl}}
\newcommand*{\rReflE}{\rRefl\rE}

\newcommand*{\judge}{\rn{Judgement}}
\newcommand*{\empRelRule}{\rn{EmptyRelation}}
\newcommand*{\empRelE}{\empRelRule\rE}
\newcommand*{\relation}{\rn{Relation}}
\newcommand*{\relationE}{\relation\rE}
\newcommand*{\choice}{\rn{Choice}}
\newcommand*{\choiceE}{\choice\rE}
\newcommand*{\product}{\rn{Product}}
\newcommand*{\productE}{\product\rE}
\newcommand*{\setop}{\rn{SetOp}}
\newcommand*{\setopE}{\setop\rE}
%\newcommand*{\diff}{\rn{SetDifference}}
%\newcommand*{\diffE}{\diff\rE}
\newcommand*{\prj}{\rn{Project}}
\newcommand*{\prjE}{\prj\rE}
\newcommand*{\sel}{\rn{Select}}
\newcommand*{\selE}{\sel\rE}
%\newcommand*{\rel}{\rn{RelationRef}}
%\newcommand*{\relE}{\rel\rE}

\newcommand*{\rC}{\rSep\rn{C}}
\newcommand*{\bool}{\rn{Boolean}}
\newcommand*{\boolC}{\bool\rC}
\newcommand*{\nott}{\rn{Neg}}
\newcommand*{\notC}{\nott\rC}
\newcommand*{\conj}{\rn{Conjunction}}
\newcommand*{\conjC}{\conj\rC}
\newcommand*{\disj}{\rn{Disjunction}}
\newcommand*{\disjC}{\disj\rC}
\newcommand*{\attVal}{\rn{AttOptVal}}
\newcommand*{\attValC}{\attVal\rC}
\newcommand*{\attAtt}{\rn{AttOptAtt}}
\newcommand*{\attAttC}{\attAtt\rC}
\newcommand*{\choiceC}{\choice\rC}


%\newcommand*{\rC}{\rSep\rn{C}}
\newcommand*{\rChoice}{\rn{Choice}\rSep}
\newcommand*{\rDist}{\rSep\rn{Dist}}

\newcommand*{\rOpDistC}[1]{\rChoice\rn{#1}\rDist\rC}
\newcommand*{\rSelDistC}{\rOpDistC {Sel}}
\newcommand*{\rPrjDistC}{\rOpDistC {Prj}}
\newcommand*{\rProdDistC}{\rOpDistC {Prod}}

\newcommand*{\rOpDist}[1]{\rChoice\rn{#1}\rDist}
\newcommand*{\rSelDist}{\rOpDist {Sel}}
\newcommand*{\rPrjDist}{\rOpDist {Prj}}
\newcommand*{\rProdDist}{\rOpDist {Prod}}


%intro spl examples
%\newcommand*{\ve}{\OB{\mathit{V}}}
%\newcommand*{\x}{\OB{\mathit{X}}}
%\newcommand*{\y}{\OB{\mathit{Y}}}
%\newcommand*{\msg}{\OB{\mathit{message}}}
%\newcommand*{\sender}{\OB{\mathit{sender}}}
%\newcommand*{\mdate}{\OB{\mathit{date}}}
%\newcommand*{\sbj}{\OB{\mathit{subject}}}
%\newcommand*{\rec}{\OB{\mathit{recepientInfo}}}
%\newcommand*{\rid}{\OB{\mathit{rid}}}
%\newcommand*{\rtype}{\OB{\mathit{rtype}}}
%\newcommand*{\mbody}{\OB{\mathit{body}}}
%\newcommand*{\encrypt}{\OB{\mathit{encrypt}}}
%\newcommand*{\decrypt}{\OB{\mathit{decrypt}}}
%\newcommand*{\encryption}{\OB{\mathit{encryption}}}
%\newcommand*{\decryption}{\OB{\mathit{decryption}}}
%\newcommand*{\employee}{\OB{\mathit{employee}}}
%\newcommand*{\signaturetab}{\OB{\mathit{signature}}}
%\newcommand*{\verification}{\OB{\mathit{verification}}}
%\newcommand*{\sign}{\OB{\mathit{sign}}}
%\newcommand*{\signed}{\OB{\mathit{signed}}}
%\newcommand*{\signKey}{\OB{\mathit{signKey}}}
%\newcommand*{\verify}{\OB{\mathit{verify}}}
%\newcommand*{\isVerified}{\OB{\mathit{isVerified}}}
%\renewcommand*{\mid}{\OB{\mathit{mid}}} %%FYI: you're defining \mid somewhere else, find where!!!!!!!!!!!!!!
%\newcommand*{\isEncrypted}{\OB{\mathit{isEncrypted}}}
%\newcommand*{\encryptionKey}{\OB{\mathit{encryptionKey}}}
%\newcommand*{\ID}{\OB{\mathit{ID}}}
%\newcommand{\faN}{\OB{\mathit{fatherName}}}
%\newcommand{\gender}{\OB{\mathit{gender}}}


% enron email system example names
\newcommand{\enron}{\OB{\mathit{en}}}
\newcommand{\eid}{\OB{\mathit{eid}}}
\newcommand{\emailid}{\OB{\mathit{email\_id}}}
\newcommand{\folder}{\OB{\mathit{folder}}}
\newcommand{\status}{\OB{\mathit{status}}}
\newcommand{\verificationkey}{\OB{\mathit{verification\_key}}}
\newcommand{\publickey}{\OB{\mathit{public\_key}}}
\newcommand{\midatt}{\OB{\mathit{mid}}}
\newcommand{\sender}{\OB{\mathit{sender}}}
\newcommand{\dateatt}{\OB{\mathit{date}}}
\newcommand{\messageid}{\OB{\mathit{message\_id}}}
\newcommand{\subject}{\OB{\mathit{subject}}}
\newcommand{\body}{\OB{\mathit{body}}}
\newcommand{\issigned}{\OB{\mathit{is\_signed}}}
\newcommand{\isencrypted}{\OB{\mathit{is\_encrypted}}}
\newcommand{\isfromremailer}{\OB{\mathit{is\_from\_remaler}}}
\newcommand{\issystemnotification}{\OB{\mathit{is\_system\_notification}}}
\newcommand{\isautoresponse}{\OB{\mathit{is\_autoresponse}}}
\newcommand{\isforwardmsg}{\OB{\mathit{is\_forward\_msg}}}
\newcommand{\rid}{\OB{\mathit{rid}}}
\newcommand{\rtype}{\OB{\mathit{rtype}}}
\newcommand{\rvalue}{\OB{\mathit{rvalue}}}
\newcommand{\rfid}{\OB{\mathit{rfid}}}
\newcommand{\reference}{\OB{\mathit{reference}}}
\newcommand{\forwardaddr}{\OB{\mathit{forwardaddr}}}
\newcommand{\pseudonym}{\OB{\mathit{pseudonym}}}
\newcommand{\suffix}{\OB{\mathit{suffix}}}
\newcommand{\username}{\OB{\mathit{username}}}
\newcommand{\mailhost}{\OB{\mathit{mailhost}}}
\newcommand{\nickname}{\OB{\mathit{nickname}}}
\newcommand{\emailAtt}{\OB{\mathit{email}}}
\newcommand{\employees}{\OB{\mathit{employeelist}}}
\newcommand{\messages}{\OB{\mathit{messages}}}
\newcommand{\recipientinfo}{\OB{\mathit{recipientinfo}}}
\newcommand{\referenceinfo}{\OB{\mathit{referenceinfo}}}
\newcommand{\automsg}{\OB{\mathit{auto\_msg}}}
\newcommand{\forwardmsg}{\OB{\mathit{forward\_msg}}}
\newcommand{\remailmsg}{\OB{\mathit{remail\_msg}}}
\newcommand{\filtermsg}{\OB{\mathit{filter\_msg}}}
\newcommand{\alias}{\OB{\mathit{alias}}}
\newcommand{\id}{\OB{\mathit{id}}}
\newcommand{\scores}{\OB{\mathit{SATscores}}}
%\newcommand{\midatt}{\OB{\mathit{mid}}}
\newcommand{\midCond}{\OB{\midatt = \xvalue}}
\newcommand{\header}{\OB{\mathit{header}}}
\newcommand{\headerAtts}{\OB{\mathit{header\_attributes}}}
\newcommand{\verifiedAt}{\OB{\mathit{verified\_at}}}
\newcommand{\signedBy}{\OB{\mathit{signed\_by}}}
\newcommand{\recievedBy}{\OB{\mathit{recieved\_by}}}
\newcommand{\verStat}{\OB{\mathit{verification\_status}}}
\newcommand{\shouldFilter}{\OB{\mathit{should\_filter}}}

\newcommand{\addressbook}{\OB{\mathit{addressbook}}}
\newcommand{\signature}{\OB{\mathit{signature}}}
\newcommand{\encryption}{\OB{\mathit{encryption}}}
\newcommand{\autoresponder}{\OB{\mathit{autoresponder}}}
\newcommand{\forwardmessages}{\OB{\mathit{forwardmessages}}}
\newcommand{\remailmessage}{\OB{\mathit{remailmessage}}}
\newcommand{\filtermessages}{\OB{\mathit{filtermessages}}}
\newcommand{\basicq}{\OB{\mathit{basic}}}
%\newcommand{\mailhost}{\OB{\mathit{mailhost}}}

%query names
\newcommand{\xvalue}{\OB{\mathit{X}}}
\newcommand{\temp}{\OB{\mathit{temp}}}
\newcommand{\tempOne}{\OB{\mathit{temp1}}}
\newcommand{\tempTwo}{\OB{\mathit{temp2}}}
\newcommand{\eq}{\OB{\mathit{empQ}}}
\newcommand{\nfq}{\OB{\mathit{Q}}}
\newcommand{\ntemp}{\OB{\mathit{temp_\mathit{enron}}}}
\newcommand{\deptOne}{\OB{\mathit{``d001"}}}
\newcommand{\mrtable}{\OB{\mathit{mes\_rec}}}
\newcommand{\mretable}{\OB{\mathit{mes\_rec\_emp}}}

% some specific queries
\newcommand{\Qbasic}{\OB{\nfq_\basicq}}
\newcommand{\Qfilter}{\OB{\nfq_\mathit{filter}}}
\newcommand{\Qforward}{\OB{\nfq_\mathit{forward}}}
\newcommand{\Qencrypt}{\OB{\nfq_\mathit{encrypt}}}
\newcommand{\Qsig}{\OB{\nfq_\mathit{sig}}}
\newcommand{\Qbf}{\OB{\nfq_\mathit{bf}}}
\newcommand{\Qsf}{\OB{\nfq_\mathit{sf}}}
\newcommand{\Qef}{\OB{\nfq_\mathit{ef}}}

\newcommand{\addressbooku}{{ADDRESSBOOK}}
\newcommand{\signatureu}{SIGNATURE}
\newcommand{\encryptionu}{ENCRYPTION}
\newcommand{\autoresponderu}{AUTORESPONDER}
\newcommand{\forwardmsgu}{FORWARDMESSAGES}
\newcommand{\remailmsgu}{REMAILMESSAGE}
\newcommand{\filtermsgu}{FILTERMESSAGES}
\newcommand{\mailhostu}{MAILHOST}

\newcommand{\addressbookl}{{ADDRESSBOOK}}
\newcommand{\signaturel}{SIGNATURE}
\newcommand{\encryptionl}{ENCRYPTION}
\newcommand{\autoresponderl}{AUTORESPONDER}
\newcommand{\forwardmsgl}{FORWARDMESSAGES}
\newcommand{\remailmsgl}{REMAILMESSAGES}
\newcommand{\filtermsgl}{FILTERMESSAGES}
\newcommand{\mailhostl}{MAILHOST}

\newcommand{\faddressbook}{\textcolor{blue} {\OB{\mathit{addressbook}}}}
\newcommand{\fsignature}{\textcolor{blue} {\OB{\mathit{signature}}}}
\newcommand{\fencryption}{\textcolor{blue} {\OB{\mathit{encryption}}}}
\newcommand{\fautoresponder}{\textcolor{blue} {\OB{\mathit{autoresponder}}}}
\newcommand{\fforwardmsg}{\textcolor{blue} {\OB{\mathit{forwardmessages}}}}
\newcommand{\fremailmsg}{\textcolor{blue} {\OB{\mathit{remailmessage}}}}
\newcommand{\ffiltermsg}{\textcolor{blue} {\OB{\mathit{filtermessages}}}}
\newcommand{\fmailhost}{\textcolor{blue} {\OB{\mathit{mailhost}}}}

\newcommand{\addressbookf}{\OB{\mathit{addressbook}}}
\newcommand{\signaturef}{\OB{\mathit{signature}}}
\newcommand{\encryptionf}{\OB{\mathit{encryption}}}
\newcommand{\autoresponderf}{\OB{\mathit{autoresponder}}}
\newcommand{\forwardmsgf}{\OB{\mathit{forwardmessages}}}
\newcommand{\remailmsgf}{\OB{\mathit{remailmessage}}}
\newcommand{\filtermsgf}{\OB{\mathit{filtermessages}}}
\newcommand{\mailhostf}{\OB{\mathit{mailhost}}}

\newcommand{\vemployees}{\OB{\mathit{v\_employeelist}}}
\newcommand{\vmessages}{\OB{\mathit{v\_messages}}}
\newcommand{\vrecipientinfo}{\OB{\mathit{v\_recipientinfo}}}
\newcommand{\vreferenceinfo}{\OB{\mathit{v\_referenceinfo}}}
\newcommand{\vautomsg}{\OB{\mathit{v\_auto\_msg}}}
\newcommand{\vforwardmsg}{\OB{\mathit{v\_forward\_msg}}}
\newcommand{\vremailmsg}{\OB{\mathit{v\_remail\_msg}}}
\newcommand{\vfiltermsg}{\OB{\mathit{v\_filtermsg}}}
\newcommand{\vmailhost}{\OB{\mathit{v\_mailhost}}}
\newcommand{\valias}{\OB{\mathit{v\_alias}}}
\newcommand{\midvalue}{\OB{\mathit{X}}}

% table values
\newcommand{\tru}{\OB{\mathit{true}}}
\newcommand{\fls}{\OB{\mathit{false}}}
\newcommand{\NULL}{\OB{\text{NULL}}}

%features
\newcommand*{\fName}{\OB{f}}
\newcommand*{\fSet}{\OB{\mathbf{F}}}
\renewcommand*{\dimMeta}{\OB{e}}
\newcommand*{\fModel}{\OB{m}}
\newcommand*{\ffSet}{\OB{\mathbf{E}}}
\newcommand*{\bTag}{\OB{b}}
\newcommand*{\bSet}{\mathbf{B}}
\newcommand*{\config}{\OB{c}}
\newcommand*{\confSet}{\OB{\mathbf{C}}}
\newcommand*{\fct}[1][\config]{\OB{\dimMeta_{#1}^t}}
\newcommand*{\vctx}{\OB{\dimMeta}}

% example feature, feature category and relation names
%\newcommand*{\France}{\OB{\mathit{France}}}
%\newcommand*{\Iceland}{\OB{\mathit{Iceland}}}
%\newcommand*{\US}{\OB{\mathit{US}}}
%\newcommand*{\Iran}{\OB{\mathit{Iran}}}
%\newcommand*{\countryF}{\OB{\mathit{country}}}
%
%\newcommand*{\fType}{\OB{\mathit{type}}}
%\newcommand*{\bank}{\OB{\mathit{bank}}}
%\newcommand*{\stock}{\OB{\mathit{stock}}}
%\newcommand*{\invest}{\OB{\mathit{invest}}}
%
%\newcommand*{\deposit}{\OB{\mathit{deposit}}}
%\newcommand*{\account}{\OB{\mathit{account}}}
%\newcommand*{\amount}{\OB{\mathit{amount}}}
%\newcommand*{\depositor}{\OB{\mathit{depositor}}}
%
%\newcommand*{\member}{\OB{\mathit{member}}}
%\newcommand*{\id}{\OB{\mathit{ID}}}
%\newcommand*{\fN}{\OB{\mathit{firstName}}}
%\newcommand*{\mN}{\OB{\mathit{middleName}}}
%\newcommand*{\lN}{\OB{\mathit{lastName}}}

%spj objects
\newcommand*{\pDB}{\OB{\underline{\db}}}
\newcommand*{\pQ}{\OB{\underline{\q}}}
\newcommand*{\pQSet}{\OB{\underline{\qSet}}}
\newcommand*{\pAtt}{\OB{\underline{\att}}}
\newcommand*{\pAttList}{\OB{\underline{\attList}}}
\newcommand*{\pRel}{\OB{\underline{\rel}}}
\newcommand*{\pRelSch}{\OB{\underline{\relSch}}}
\newcommand*{\pRelCont}{\OB{\underline{\relCont}}}
\newcommand*{\pRelContSet}{\OB{\underline{\relContSet}}}
\newcommand*{\pSch}{\OB{\underline{\sch}}}
\newcommand*{\pInst}{\OB{\underline{\dbInst}}}
\newcommand*{\pInstSet}{\OB{\underline{\dbInstSet}}}
\newcommand*{\pAttSet}{\OB{\underline{\attSet}}}
\newcommand*{\pRelSet}{\OB{\underline{\relSet}}}
\newcommand*{\pRelSchSet}{\OB{\underline{\relSchSet}}}
\newcommand*{\pSchSet}{\OB{\underline{\schSet}}}
\newcommand*{\pAttOpCte}{\OB{\op \pAtt \cte}}
\newcommand*{\pAttOpAtt}{\OB{\op {\pAtt_1}  {\pAtt_2}}}
\newcommand*{\pElem}{\OB{\underline \elem}}
\newcommand*{\pTuple}{\OB{\underline \tuple}}
\newcommand*{\pTab}{\OB{\underline \tab}}


%sql objects
\newcommand*{\sqlQ}{\OB{\underline{sql}}}

%symbols
\newcommand*{\db}{\OB{D}}
\newcommand*{\q}{\OB{q}}
\newcommand*{\att}{\OB{a}}
\newcommand*{\rel}{\OB{r}}
\newcommand*{\tab}{\OB{t}}
\newcommand*{\relSch}{\OB{s}}
\newcommand*{\relCont}{\OB{U}}
\newcommand*{\attList}{\OB{A}}
\newcommand*{\sch}{\OB{S}}
\newcommand*{\dbInst}{\OB{\mathcal{I}}}
\newcommand*{\tuple}{\OB{u}}
\newcommand*{\vTuple}{\tuple}
\newcommand*{\annot}[2][\dimMeta]{\OB{{#2}^{#1}}}
\newcommand*{\elem}{\OB{x}}
\newcommand*{\elemSet}{\OB{X}}
\newcommand*{\cond}{\OB{\theta}}
\newcommand*{\pset}{\OB{\underline {X}}}



%set symbols
\newcommand*{\qSet}{\OB{\mathbf{Q}}}
\newcommand*{\dbInstSet}{\OB{\mathbf{I}}}
\newcommand*{\attSet}{\OB{\mathbf{A}}}
\newcommand*{\tabSet}{\OB{\mathbf{T}}}
\newcommand*{\schSet}{\OB{\mathbf{\mathcal{S}}}}
\newcommand*{\relSchSet}{\OB{\mathbf{S}}}
\newcommand*{\relContSet}{\OB{\mathbf{R}}}
\newcommand*{\condSet}{\OB{\mathbf{\Theta}}}


% vspj objects
\newcommand*{\vDB}{\db}
\newcommand*{\vSchDef}{\OB{\setDef {{\vRelSch_1 , \ldots, \vRelSch_n}}^\fModel}}
%{\{\widetilde{s}_{R_1}^{F_1}, \ldots, \widetilde{s}_{R_n}^{F_n}\}^\dimMeta}}
\newcommand*{\vdbInst}{\OB{\dbInst}}
\newcommand*{\vdbInstSet}{\OB{\dbInstSet}}
\newcommand*{\vObj}{\OB{o}}
\newcommand*{\vObjSet}{\OB{\mathbf{O}}}
\newcommand*{\vQ}{\OB{\q}}
%\newcommand*{\vRelSch}[1][\pRel]{\OB{{\VVList{s}}_{#1}}}
\newcommand*{\vRelSch}{\OB{\relSch}}
\newcommand*{\vRelCont}{\OB{\relCont}}
\newcommand*{\vSch}{\OB{\sch}}
%\newcommand*{\vSch}{\OB{\VVSet {\mathcal{S}}}}
\newcommand*{\vset}{\OB{X}}
\newcommand*{\vTab}{\tab}
\newcommand*{\vRel}{\rel}
\newcommand*{\vAtt}{\att}
\newcommand*{\nameVar}{\OB{n}}

%\makeatletter
%\def\vRel{%
%   \@ifnextchar[%
%     {\vRel@i}
%     {\vRel@i[\dimMeta]}%
%}
%\def\vRel@i[#1]{%
%   \@ifnextchar[%
%     {\vRel@ii{#1}}
%     {\vRel@ii{#1}[\pRel]}%
%}
%\def\vRel@ii#1[#2]{%
%\OB{{#2}^{#1}}
%}
%\makeatother

\newcommand*{\relNum}{\OB{n}}

\makeatletter
\def\vRelConfed{%
   \@ifnextchar[%
     {\vRelConfed@i}
     {\vRelConfed@i[\config]}%
}
\def\vRelConfed@i[#1]{%
   \@ifnextchar[%
     {\vRelConfed@ii{#1}}
     {\vRelConfed@ii{#1}[\pRel]}%
}
\def\vRelConfed@ii#1[#2]{%
\OB{{#2}^{#1}}
}
\makeatother

\makeatletter
\def\optAtt{%
   \@ifnextchar[%
     {\optAtt@i}
     {\optAtt@i[\dimMeta]}%
}
\def\optAtt@i[#1]{%
   \@ifnextchar[%
     {\optAtt@ii{#1}}
     {\optAtt@ii{#1}[\vAtt]}%
}
\def\optAtt@ii#1[#2]{%
%\OB{{\widetilde{#2}}^{#1}}
\OB{\annot[#1]{#2}}
}
\makeatother

%\newcommand*{\vAttt}[2]
%\newcommand*{\attPres}{\OB{F}}
\newcommand*{\vAttList}{\OB{\attList}}
%\widetilde{l}}}
%\newcommand*{\vRelDef}{\OB{\left[\pRel^{\dimMeta}:\vAttList\right]}}

\makeatletter
\def\vRelDef{
   \@ifnextchar[
      {\vRelDef@i}
      {\vRelDef@i[\vRel]}
}
\def\vRelDef@i[#1]{
   \@ifnextchar[
      {\vRelDef@ii{#1}}
      {\vRelDef@ii{#1}[\dimMeta]}
}
\def\vRelDef@ii#1[#2]{
   \@ifnextchar[
      {\vRelDef@iii{#1}{#2}}
      {\vRelDef@iii{#1}{#2}[\vAttList]}
}
\def\vRelDef@iii#1#2[#3]{
\OB{{#1}\left({#3}\right)^{#2}}
}

\newcommand*{\vRelDefNum}[1]{\OB{\vRelDef [\vRel_{#1}] [\dimMeta_{#1}] [\vAttList_{#1}] }}
\newcommand*{\vRelDefNumF}[2]{\OB{\vRelDef [\vRel_{#1}][\vAttList_{#1}][\dimMeta \wedge \dimMeta_{#1}]}}

%\newcommand*{\vRelDef}{\OB{\left[\pRel^{\dimMeta}:\vAttList\right]}}
\newcommand*{\vRelConf}[1][\vRel]{\OB{{#1}^\config}}
\newcommand*{\vAttSet}{\OB{\attSet}}
\newcommand*{\vTabSet}{\OB{\tabSet}}
\newcommand*{\vRelSchSet}{\OB{\relSchSet}}
\newcommand*{\vRelContSet}{\OB{\relContSet}}
\newcommand*{\vSchSet}{\OB{\schSet}}
\newcommand*{\vInstSet}{\OB{\dbInstSet}}
\newcommand*{\empSet}{\OB{\varnothing}}
\newcommand*{\emp}{\OB{\varepsilon}}
\newcommand*{\empAtt}{\OB{\varepsilon}}
\newcommand*{\empRel}{\OB{\varepsilon}}
\newcommand*{\vAttOpCte}{\OB{\op \pAtt \cte}}
\newcommand*{\vAttOpAtt}{\OB{\op {\pAtt_1} {\pAtt_2}}}
\renewcommand*{\tag}[2]{\OB{{#1}^{#2}}}
\newcommand*{\pc}{\OB{pc}}
\newcommand*{\getPC}[1]{\OB{pc\left(#1\right)}}
\newcommand*{\getRel}[1]{\OB{rel\left(#1\right)}}
\newcommand*{\getAtt}[1]{\OB{att\left(#1\right)}}

%conditions
\newcommand*{\pCond}{\OB{\underline{\cond}}}
\newcommand*{\cte}{\OB{k}}
\newcommand*{\pCondSet}{\OB{\underline{\condSet}}}

% variational conditions
\newcommand*{\vCond}{\OB{\cond}}
\newcommand*{\vCondSet}{\OB{\condSet}}

% vspj typing
\newcommand*{\vType}{\OB{\vAttList}}
%\newcommand*{\vContext}{\OB{\vctx}} % --> remove ti

% trans part(Qiaoran)
%\newcommand*{\vSql}{\OB{vq}}
%\newcommand*{\vSqlSet}{\OB{\mathbf{VQ}}}
%\newcommand*{\vRelList}{\OB{rl}}
%\newcommand*{\vRelListSet}{\OB{\mathbf{VrSet}}}
%\newcommand*{\oAttList}{\OB{al}}
%\newcommand*{\oAttListSet}{\OB{\mathbf{Val}}}
%\newcommand*{\vSqlselect}{\textbf{Select} (\vAttList, \vRelList, \vCond)}

% Qiaoran add for translation part
% \newcommand*{\tqS}{\OB{T_Q}}
%\newcommand*{\tqTrans}[1]{\OB{\tqS({#1})}}
%\newcommand*{\trS}{\OB{T_R}}
%\newcommand*{\trTrans}[1]{\OB{\trS({#1})}}

%variational list and set
\newcommand*{\listl}{\OB{l}}
\newcommand*{\sets}{\OB{s}}
\newcommand{\VSet}[1]{\smash{\accentset{\rightarrow}{#1}}}
\newcommand{\VVSet}[1]{\smash{\accentset{\hookrightarrow}{#1}}}
\newcommand{\VList}[1]{\smash{\accentset{\rightarrowtriangle}{#1}}}
\newcommand{\VVList}[1]{\smash{\accentset{\lhook\joinrel\rightarrowtriangle}{#1}}}
\newcommand{\VMap}[1]{\smash{\accentset{\mapsto}{#1}}}

\newcommand{\blueit}[1]{\textcolor{blue} {#1}}

\newcommand{\Cat}[1]{\mathit{#1}}
\newcommand{\myOR}{\hspace{1.5ex}|\hspace{1.5ex}}
\newcommand{\VVal}[1]{#1'}
\newcommand{\VVVal}[1]{#1''}

\newcommand{\vlift}[1]{\lceil #1 \rceil_S^{S'}}


% defining syntax
\newcommand{\synDef}[2]{\OB{#1 \in #2 }}
\newcommand{\eqq}{\OB{\Coloneqq}}
\renewcommand{\t}{\OB{\prog{true}}}
\newcommand{\f}{\OB{\prog{false}}}

%constraining function
\newcommand{\constrain}[2][\vSch]{\OB{\lfloor #2 \rfloor_{#1}}}

% defining semantics
\newcommand*{\fS}{\OB{\mathbb{E}}}
\newcommand*{\fSem}[2][\config]{\OB{\fS\sem[#1]{#2}}}

\newcommand*{\elemS}{\OB{\mathbb{X}}}
\newcommand*{\elemSem}[2][\config]{\OB{\elemS\sem[#1]{#2}}}

\newcommand*{\elemG}{\OB{\mathcal{G}}}
\newcommand*{\elemGroup}[1][\elem]{\OB{\elemG\left({#1}\right)}}

\newcommand*{\qG}{\OB{\mathcal{Q}}}
\newcommand*{\qGroup}[1]{\OB{\qG\left({#1}\right)}}

\newcommand*{\cG}{\OB{\mathcal{C}}}
\newcommand*{\cGroup}[1][\vCond]{\OB{\cG\left({#1}\right)}}

\newcommand*{\aG}{\OB{\mathcal{A}}}
\newcommand*{\aGroup}[1][\vAttList]{\OB{\aG\left({#1}\right)}}

\newcommand*{\oS}{\OB{\mathbb{O}}}
\newcommand*{\oSem}[2][\config]{\OB{\oS\sem[#1]{#2}}}

\newcommand*{\olS}{\OB{\mathbb{A}}}
\newcommand*{\olSem}[2][\config]{\OB{\olS\sem[#1]{#2}}}

\newcommand*{\orS}{\OB{\mathbb{R}}}
\newcommand*{\orSem}[2][\config]{\OB{\orS\sem[#1]{#2}}}

\newcommand*{\otS}{\OB{\mathbb{T}}}
\newcommand*{\otSem}[2][\config]{\OB{\otS\sem[#1]{#2}}}

\newcommand*{\ouS}{\OB{\mathbb{U}}}
\newcommand*{\ouSem}[2][\config]{\OB{\ouS\sem[#1]{#2}}}

\newcommand*{\ovS}{\OB{\mathbb{V}}}
\newcommand*{\ovSem}[2][\config]{\OB{\ovS\sem[#1]{#2}}}

\newcommand*{\osS}{\OB{\mathbb{S}}}
\newcommand*{\osSem}[2][\config]{\OB{\osS\sem[#1]{#2}}}

\newcommand*{\odbS}{\OB{\mathbb{I}}}
\newcommand*{\odbSem}[2][\config]{\OB{\odbS\sem[#1]{#2}}}

%\newcommand*{\eS}{\OB{Q}}
%\newcommand*{\eSem}[2][\config]{\OB{\eS\sem[#1]{#2}}}

\newcommand*{\ecS}{\OB{\mathbb{C}}}
\newcommand*{\ecSem}[2][\config]{\OB{\ecS\sem[#1]{#2}}}

\newcommand*{\eeS}{\OB{\mathbb{Q}}}
\newcommand*{\eeSem}[2][\config]{\OB{\eeS\sem[#1]{#2}}}

%\newcommand*{\tqS}{\OB{T_Q}}
%\newcommand*{\tqSem}[1]{\OB{\tqS\sem[]{#1}}}

% vset operations
\newcommand{\pushIn}[1]{\OB{\downarrow\!\!\left(#1\right)}}

% spj operations
\newcommand{\pPrj}[2][\pAtt]{\OB{\pi_{#1} #2}}
\newcommand{\pSel}[2][\pCond]{\OB{\sigma_{#1} #2}}
\newcommand{\setDef}[1]{\OB{\{#1\}}}

% conditions
\newcommand{\op}[2]{\OB{#1\bullet#2}}
\newcommand{\annd}[1]{\OB{{#1}_1 \wedge {#1}_2}}
\newcommand{\orr}[1]{\OB{{#1}_1 \vee {#1}_2}} 

% vspj operations
\newcommand{\vPrj}[2][\vAttList]{\OB{\pi_{#1} #2}}
\newcommand{\vSel}[2][\vCond]{\OB{\sigma_{#1} {#2}}}
\newcommand{\vRen}[2][\nameVar]{\OB{\rho_{#1} #2}}
\newcommand{\project}[2]{\OB{\pi_{#1} \left({#2}\right)}}
\newcommand{\projectRel}[2]{\OB{\pi_{#1} {#2}}}
\newcommand{\select}[2]{\OB{\sigma_{#1} \left({#2}\right)}}
\newcommand{\selectRel}[2]{\OB{\sigma_{#1} {#2}}}
\newcommand{\oatt}[2]{\OB{{#1}^{#2}}}
\newcommand{\join}[3]{\OB{\left({#1}\right) \bowtie_{#3} \left({#2}\right)}}
\newcommand{\joinRelL}[3]{\OB{{#1} \bowtie_{#3} \left({#2}\right)}}
\newcommand{\joinRelR}[3]{\OB{\left({#1}\right) \bowtie_{#3} {#2}}}
\newcommand{\joinRel}[3]{\OB{{#1} \bowtie_{#3} {#2}}}
\newcommand{\rename}[2]{\OB{\rho_{#1}  \left({#2}\right)}}
%\newcommand{\choiice}[3]{\OB{\chc [\left({#1}\right)] {\left({#2}\right), \left({#3}\right)}}}
\newcommand{\choiice}[3]{\OB{\chc [\left({#1}\right)] {{#2}, {#3}}}}
%\newcommand{\choiceL}[3]{\OB{\chc [\left({#1}\right)] {{#2}, \left({#3}\right)}}}
%\newcommand{\choiceR}[3]{\OB{\chc [\left({#1}\right)] {\left({#2}\right), {#3}}}}
%\newcommand{\choiceS}[3]{\OB{\chc [{#1}] {\left({#2}\right), \left({#3}\right)}}}
\newcommand{\choiceS}[3]{\OB{\chc [{#1}] {{#2}, {#3}}}}
%\newcommand{\choiceSL}[3]{\OB{\chc [{#1}] {{#2}, \left({#3}\right)}}}
%\newcommand{\choiceSR}[3]{\OB{\chc [{#1}] {\left({#2}\right), {#3}}}}
%\newcommand{\chc}[2][\dimMeta]{\OB{#1\chcL#2\chcR}}


% query type environment
%\newcommand{\env}[4][\vctx][\vSch]{\OB{#1,#2 \vdash#3:#4}}
\makeatletter
\def\env{
  \@ifnextchar[
    {\env@i}
    {\env@i[\vctx]}
}
\def\env@i[#1]{
  \@ifnextchar[
    {\env@ii{#1}}
    {\env@ii{#1}[\vSch]}
}
\def\env@ii#1[#2]#3#4{
{\OB{#1,#2 \vdash#3:#4}}
}
\makeatother

\newcommand{\pEnv}[3][\pSch]{\OB{#1 \vdash #2 : #3}}
\newcommand{\envWithSchema}[2][\OB{\vctx}]{\env[#1]{\vRel}{#2}}
\newcommand{\envRelInSch}[2][\vSch]{\env[#1]{#2}{\vType}}
\newcommand{\envOne}[1][\vctx]{\env[#1]{\vQ_1}{\envInContext [\vctx_1] \vType_1}}
\newcommand{\envTwo}[1][\vctx]{\env[#1]{\vQ_2}{\envInContext [\vctx_2] \vType_2}}
\newcommand{\envPrime}{\env{\vQ}{\envInContext[\VVal \vctx]{\VVal \vType}}}

% condition type environment
\newcommand{\envCond}[2][\vctx, \vType]{\OB{#1\vdash #2}}
\newcommand{\envCondAnnot}[2][\vctx, {\pushIn{\annot [\VVal \vctx] \vType}}]{\OB{#1\vdash #2}}

% type and variation context
\newcommand{\envL}{\OB{\lfloor}}
\newcommand{\envR}{\OB{\rfloor}}
\newcommand{\envInContext}[2][\vctx]{\OB{{#2}^{#1}}}
\newcommand{\envEval}[2]{#1 \equiv #2}
\newcommand{\imply}{\OB{\rightarrow}}
\newcommand{\subsume}[2]{\OB{#1 \prec #2}}
\newcommand{\subsumeExpl}[2]{\OB{#1 \preccurlyeq #2}} 
\newcommand{\nsubsume}[2]{\OB{#1 \not \prec #2}}

% object in a variational context
\newcommand{\relInContext}[2][\vRel]{\OB{#1^{#2}}}
\newcommand{\attInContext}[2][\vAtt]{\OB{#1^{#2}}}
\newcommand{\defType}[2][\vType]{\OB{#2: #1}}


% satisfiability function
\newcommand{\sat}[1]{\OB{\mathit{sat}\left(#1\right)}}
\newcommand{\taut}[1]{\OB{\mathit{taut}\left(#1\right)}}

% dbms
\renewcommand{\dom}[2][\vdbInst]{\OB{\mathit{dom}_{#1}\left(#2\right)}}
\newcommand{\type}[1][\pAtt]{\OB{\mathit{type}\left(#1\right)}}
\newcommand{\attr}[1][\pRel]{\OB{\mathit{attr}\left(#1\right)}}
\newcommand{\arity}[1][\pRel]{\OB{\mathit{arity}\left(#1\right)}}
\newcommand{\feat}[2][\vObj]{\OB{\mathit{presCond_{#1}}\left(#2\right)}}
\newcommand{\obj}[1]{\OB{\mathit{obj}\left(#1\right)}}

%\newcommand{\chc}[2][\dimMeta]{\OB{#1\chcL#2\chcR}}
%\newcommand{\chcPP}[3][\dimMeta]{\chc[#1]{\prog{#2},\prog{#3}}}
%\newcommand{\chcPPP}[4][\dimMeta]{\chc[#1]{\prog{#2},\prog{#3},\prog{#4}}}

% special equalities
\newcommand{\spcEq}[1][0cm]{\OB{\hspace{#1}=}}
\newcommand{\spcEquiv}[1][0cm]{\OB{\hspace{#1}\equiv}}


% operations w.r.t. c
\newcommand{\equivc}{\OB{\equiv_{\config}}}


% enumeration and indexing 
\newcommand{\vvn}[2]{\OB{{#1_1}^{#2_1}, \ldots, {#1_n}^{#2_n}}}

% formulation in general
\newcommand{\paran}[1]{\OB{\left(#1\right)}}



\newcommand*{\natNum}{\ensuremath{\mathbb{N}}}
\newcommand*{\natStar}{\ensuremath{\mathbb{N}^\ast}}


\begin{document}
 
\title{CS 517 Project}
%\date{}
\maketitle
\centerline{\author{Parisa S. Ataei}}

\begin{comment}
%Mike's comments on the report:
Great thorough description of the language syntax etc. The description of the formulas is a good start but still confusing to me. I can't see that these formulas are doing NP stuff, but they look like they are just checking P conditions. NP is about guess+check, and it might help to think about what is guessed and what is checked in the final formula (i.e., what are the free variables)?

For inputs of different sizes, what is the size of the formula and how long does it take the solver to run? Can you find both yes and no-instances to this NP-hard problem? Where will you find/generate sample data? How does the cost of the solver scale with input size of the problem? What is the limit of feasible inputs that can be solved with this method?
\end{comment}

\section{Introduction}
\label{sec:intro}

Program equivalence is generally an undecidable problem, however,
arguably, it is easier to solve compared to program correctness although
the two have lots of similarities that one can exploit to solve one using
the other. While for program correctness one needs to explicitly define
the formal semantics of a language, equivalency rules can be defined
for a language to search the space for a given program to find its 
equivalence programs. However, for such an expensive search to be conclusive,
the set of reduction rules must be terminating and confluent.

I designed the \emph{Variational Relational Algebra (VRA)} for my research
which combines Formula Choice Calculus~\cite{HW16fosd} and relational algebra.
In this project, I reduce a program written in VRA to a propositional formula
and run a SAT solver on it to see if there exists an equivalence program 
w.r.t. the variational point. The simplified syntax of VRA is given in \secref{bg}, 
the problem formalization is given  in \secref{def}, and the reduction of
program equivalence to a propositional formula is provided in \secref{reduce}.
Finally, \secref{impl} elaborates on the implementation and \secref{disc}
discusses the assumptions made to allow a reasonable reduction of AST to
SAT formulas and how different features to the language break the reduction.
Additionally, I discuss why I only consider the structural equivalence of programs
w.r.t. the variational point instead of a more general equivalency.


\section{Background}
\label{sec:bg}

\figref{vra} defines the syntax of the simplified VRA used in this project.
Feature expressions are propositional formulas of the set of features defined
for a database. Assume we have a database with features $\fName_1$ and 
$\fName_2$ and two tables $\vRel_1 (\att_1, \att_2)$ and $\vRel (\att_1, \att_3)$. 
For simplicity, I assume the database in use is a traditional database and not 
a variational database. 
$\vRel_1$ refers to the relation $\vRel_1$ in the database and returns the table
as stored in the database. 
Projection $\vPrj \vQ$ projects a set of attributes $\vAttList$ from the subquery $\vQ$.
For example, the query $\vPrj [\att_1] \vRel_1$ projects the attribute $\att_1$ from the 
relation $\vRel_1$ stored in the database. As an example of variational query 
$\vQ_1 = \pi_{ \chc [\fName_1] {\att_1, \att_2}} \vRel_1$ projects the attribute $\att_1$ when
$\fName_1$ evaluates to \t\ and attribute $\att_2$ otherwise. Selection 
filters tuples returned from a subquery based on a condition where the condition can
also be variational. Cross product just provides the cross product of the 
two given tables. A choice of queries allows one to write queries variationally on
the top level. For example, one can write $\vQ_1$ as 
$\chc [\fName_1] {\vPrj [\att_1] \vRel_1, \vPrj [\att_2] \vRel_1}$. Note that a 
choice cannot necessarily be pushed in or out. For example, the query
$\chc [\fName_1] {\vRel_1, \vRel_2}$ cannot be simplified.
Finally, the empty relation $\empRel$ is introduced as a convenient for 
users to indicate that an alternative of a choice does not do anything while
the other alternative is only valid under the condition provided by the 
dimension of the choice (in the choice $\chc {\vQ_1, \vQ_2}$ the feature expression
\dimMeta\ is called the dimension of the choice).
Throughout this report a \emph{variational point} refers to where a choice
has been used in query, this could be at the query level, the attribute
level, or the condition level.


\begin{figure}[hbt!]
\begin{syntax}
\small
% feature expressions
\synDef{\dimMeta}{\ffSet}
  &\eqq& \multicolumn{2}{l}{%
         \t \myOR \f \myOR \fName \myOR \neg\fName
         \myOR \dimMeta\wedge\dimMeta \myOR \dimMeta\vee\dimMeta}
\\[1.5ex]

% variational conditions
\synDef{\vCond}{\vCondSet}
  &\eqq& \multicolumn{2}{l}{%
         \t \myOR \f \myOR \att\bullet\cte \myOR \att\bullet\att
         \myOR \neg\vCond \myOR \vCond\vee\vCond} \\
     &|& \multicolumn{2}{l}{\vCond\wedge\vCond \myOR \chc{\vCond,\vCond}}
\\[1.5ex]

\synDef{\vAttList}{\vAttSet}
  &\eqq&  \multicolumn{2}{l}{%
  \vAtt \myOR \chc{\vAtt,\vAtt} \myOR \vAtt,\vAttList \myOR \empRel \myOR \chc{\vAtt,\vAtt},\vAttList}
  \\[1.5ex]
  
% variational relational algebra
\synDef{\vQ}{\qSet}
  &\eqq& \vRel                 & \textit{Relation reference} \\
%     &|& \vRen[\vRel]{\vQ}     & \textit{Renaming} \\
     &|& \vPrj[\vAttList]{\vQ} & \textit{Projection} \\
     &|& \vSel\vQ              & \textit{Selection} \\
     &|& \vQ \times \vQ  & \textit{Cross product} \\
     &|& \chc{\vQ,\vQ}         & \textit{Choice} \\
     &|& \empRel               & \textit{Empty relation} \\
%    &|& \vQ \times \vQ        & \textit{Cartesian Product} \\
%    &|& \vQ \circ \vQ         & \textit{Set operation} \\
\end{syntax}

\caption{Syntax of variational relational algebra, where $\bullet$ ranges over
comparison operators ($<, \leq, =, \neq, >, \geq$), 
\cte\ over constant values,
\att\ over attribute names, and \vAttList\ over lists of variational attributes.
The syntactic category \dimMeta\ represents feature expressions, \vCond\
is variational conditions, and \vQ\ is variational queries.
}
\label{fig:vra}
\end{figure}


 

\section{Problem Definition}
\label{sec:def}

The problem I will be attacking is:
\emph{Assuming that a given a program written in VRA is type correct
and is of canonical form, is there exist an equivalent program 
with less variational points?}
A program is type correct if it passes the type system of VRA~\cite{vldbArXiv},
i.e., the query conforms to the underlying schema of the database.
The canonical form of program represent a class of programs that are 
semantically equivalent based on equivalency rules of relational algebra 
and the choices are pushed down into the query as much as possible.
A program in the canonical form follows the following
order for its operators, if they exist: projection, selection, cross-product; where
choices could appear at attribute level, condition level, and query level.
\TODO {I will provide examples of this for the final report to help the reader understand the concept better.}
This problem is similar to the minimization of binary decision diagrams~\cite{minBDD}
since it determines if a program can be written with less variational points
and thus is NP-hard.




\section{Reduction to SAT}
\label{sec:reduce}
I generate multiple SAT formulas to check:
1) if variations used in the query are valid (reasonable),
e.g., the variation in $\chc [\f] {\vQ_1, \vQ_2}$ is not reasonable 
although the query is type correct, elaborated in \secref{val},
2) if dead alternative branches exist in a choice, e.g., 
the alternative branch $\vQ_2$ will never be executed in the 
query $\chc [\fName_1] {\chc [\fName_1 \vee \fName_2] {\vQ_1, \vQ_2} \vQ_3}$, 
elaborated in \secref{dead},
and 
3) if a choice has redundancy (this relies on having a structurally equivalence
relation for the pure relational queries), e.g., the $\vQ_1$ subquery is redundant in the query 
$\chc [\dimMeta_1] {\vQ_1, \chc [\dimMeta_2] {\vQ_1, \vQ_2}}$, elaborated in \secref{red}.
Since the VRA is inductive I generate these SAT problems in a bottom-up
approach while keeping and updating an environment of variables introduced 
for the formulas.

I generate fresh variables as follows as the first step for generating each SAT formula:

\begin{itemize}
\item Attribute names are denoted by variables $\att_1, \att_2, \cdots$. Note that the empty attribute is associated with \t.
\item Conditions are denoted by variables $\cond_1, \cond_2, \cdots$ except for the variational conditions.
Note that it is easy to define semantically equivalence on pure relational conditions, thus, there is no need
to break down the conditions except for variational ones. Also, especially since we are not concern with
semantical equivalence in this project.
\item Relations are denoted by variables $\vRel_1, \vRel_2, \cdots$. And the empty relation \empRel\ is associated with \t. 
\item The same feature names are used as their variables in the generated formula. 
\item The subqueries are denoted by variables $\vQ_1, \vQ_2, \cdots$ and are generated bottom-up.
\end{itemize}



\subsection{Generating Validity SAT Formulas}
\label{sec:val}
The validity propositional formula for a given query is generated as follows:

\begin{itemize}
\item For empty relation \empRel\ generate \t.
\item For each relation \vRel\ generate the clause $\t \wedge \vRel$.
\item For each attribute \att\ generate the clause $\t \wedge \att$.
\item For the set of attributes $\att_1, \cdots, \att_n$ generate the formula
$(\t \wedge \att_1) \wedge \cdots \wedge (\t \wedge \att_n)$.
\item For each condition \cond\ generate the clause $\t \wedge \cond$.
\item For a variational attribute $\chc {\att_1, \att_2}$ generate the clause
$(\dimMeta \wedge (\t \wedge \att_1)) \vee (\neg \dimMeta \wedge (\t \wedge \att_2))$.
\item For the query $\vPrj \vQ$ generate the formula 
$\valid {\vAttList} \wedge \valid {\vQ}$ where $\valid \vAttList$ is the validity formula generated for the set 
of attributes and \valid \vQ\ is the validity formula generated for the subquery \vQ, both based 
on the description provided above.
\item For the query $\vSel \vQ$ generate the formula 
$\valid {\cond} \wedge \valid {\vQ}$ where \valid \cond\ is the validity formula generated for the condition \cond\
based on the description provided. 
\item For the query $\chc {\vQ_1, \vQ_2}$ generate the formula
$(\dimMeta \wedge \valid {\vQ_1}) \vee (\neg \dimMeta \wedge \valid {\vQ_2})$.
\end{itemize}

If the validity formula generated for a query is not satisfiable then the query contains an invalid (unreasonable) variation and can be rewritten to omit such variation.

\subsection{Generating Dead-Branch SAT Formulas}
\label{sec:dead}
The formulas to determine if a query has dead branches  are generated as follows:

\begin{itemize}
\item For every choice $x = \chc {x_1, x_2}$ in a given query 
where $x$ is a meta-variable that
ranges over syntactic categories of conditions, attributes, and queries
generate two formulas $\deadl x = \dimMeta \to \dimMeta_l$ and 
$\deadr x = \neg \dimMeta \to \dimMeta_r$ where $\dimMeta_l$
is the dimension of the choice within the left branch, i.e., $x_1$ and 
$\dimMeta_r$ is the dimension of the choice within the right branch, i.e., $x_2$.
And $l \to r$ is implication which can be substituted by $\neg l \vee r$.
\end{itemize}

Intrinsically, \deadl . and \deadr . determine if the dimension of a nested choice
is more general that its outside choice which cause some branches to never be executed.
Note that \deadl . and \deadr . are generated for each syntactic category separately.
And they solely focus on the interaction of feature expressions. Thus,
if \sat {\deadl \vQ} or \sat {\deadr \vQ}, then the query \vQ\ contains some dead alternative branches
and can be simplified.

\subsection{Generating Redundant-Branch SAT Formulas}
\label{sec:red}

To determine if a query has redundant branches formulas may be generated
for each nested choice of a specific syntactic category, assuming that we have a structurally equivalence relationship ($\equiv$) over pure relational syntactic 
category of $x$, as follows:

\begin{itemize}
\item If the choice has the format $\chc [\dimMeta_1] {x_1, \chc [\dimMeta_2] {x_2, x_3}} $ where $x_1$
does not have a top level choice we have:
\begin{itemize}
\item
If $x_1 \equiv x_2 $ then they both are assigned the same variable name, say $x^\prime$, then generate the formula\\
\centerline{$\left(\left(\dimMeta_1 \wedge x^\prime\right) \vee \left(\neg \dimMeta_1 \wedge \dimMeta_2 \wedge x^\prime\right)\right) \leftrightarrow \left(\dimMeta_1 \vee \dimMeta_2 \right) \wedge x^\prime$}.
\item
If $x_1 \equiv x_3 $ then they both are assigned the same variable name, say $x^\prime$, then generate the formula\\
\centerline{$\left(\left(\dimMeta_1 \wedge x^\prime\right) \vee \left(\neg \dimMeta_1 \wedge \neg \dimMeta_2 \wedge x^\prime\right)\right) \leftrightarrow \left(\dimMeta_1 \vee \neg \dimMeta_2 \right) \wedge x^\prime$}.\end{itemize}

\item If the choice has the format $\chc [\dimMeta_1] {\chc [\dimMeta_2] {x_1, x_2}, x_3} $where $x_3$
does not have a top level choice we have:
\begin{itemize}
\item 
If $x_2 \equiv x_3 $ then they both are assigned the same variable name, say $x^\prime$, then generate the formula\\
\centerline{$\left(\left(\neg \dimMeta_1 \wedge x^\prime\right) \vee \left( \dimMeta_1 \wedge \neg\dimMeta_2 \wedge x^\prime\right)\right) \leftrightarrow \left(\dimMeta_1 \wedge \dimMeta_2 \right) \wedge x^\prime$}.
\item
If $x_1 \equiv x_3 $ then they both are assigned the same variable name, say $x^\prime$, then generate the formula\\
\centerline{$\left(\left(\neg \dimMeta_1 \wedge x^\prime\right) \vee \left( \dimMeta_1 \wedge \dimMeta_2 \wedge x^\prime\right)\right) \leftrightarrow \left(\dimMeta_1 \wedge \neg \dimMeta_2 \right) \wedge x^\prime$}.
\end{itemize}
\item If the choice has the format $\chc [\dimMeta_1] {\chc [\dimMeta_2] {x_1, x_2}, \chc [\dimMeta_3] {x_3, x_4}} $ we have:
\begin{itemize}
\item 
If $\dimMeta_2 \equiv \dimMeta_3$ and $x_2 \equiv x_4$ (thus $x_2$ and $x_4$ are assigned the same variable $x^\prime$) generate the following formula:\\
\centerline{$\left(\left( \dimMeta_1 \wedge \neg \dimMeta_2 \wedge x^\prime\right) \vee
\left(\neg \dimMeta_1 \wedge \neg \dimMeta_3 \wedge x^\prime\right)\right) \leftrightarrow \neg \dimMeta_2 \wedge x^\prime$}
\item 
If $\dimMeta_2 \equiv \dimMeta_3$ and $x_1 \equiv x_3$ (thus $x_1$ and $x_3$ are assigned the same variable $x^\prime$) generate the following formula:\\
\centerline{$\left(\left( \dimMeta_1 \wedge  \dimMeta_2 \wedge x^\prime\right) \vee
\left(\neg \dimMeta_1 \wedge  \dimMeta_3 \wedge x^\prime\right)\right) \leftrightarrow  \dimMeta_2 \wedge x^\prime$}\end{itemize}
\end{itemize}

\rdnt \vQ denotes all the formulas generated as described above for a given query \vQ.
If at least one of these queries is a tautology then we can conclude that there exists 
a redundant branch in the query. 


\section{Implementation}
\label{sec:impl}

I am currently implementing the project in Haskell and I use the SBV library for
solving the SAT problems. The GitHub repo will be provided with the final submission.

\section{Discussion}
\label{sec:disc}
Made the mistake of reading the news and lost track of time! However, the following contains 
my initial thoughts on the project. I will provide a comprehensive discussion for the final report. 

After thinking about this reduction for a week, I do not think that, generally speaking, an AST 
can be linearized to the extent that it is converted to a single propositional formula to determine
variation minimization or even worse structural or semantical equivalence. 
Thus, after narrowing down the problem definition, I began thinking about different types of minimizations
such as redundant branches and dead branches. I also believe I was able to do so because of the
nature of choice calculus and that it uses propositional formulas within itself. At the same time, I 
am skeptical that the reductions I have provided cover all the cases and I am interested to 
know if there is a way to prove or deny that this reduction can certainly determine if
a variational query can be minimized or not.
I did not simplified the query language that much and I believe adding operations such as 
union and intersection would not break down the reduction. However, an important difference
of the language that I provided here with the original VRA is that variation appears in the language
used in this project in only one manner as opposed to original VRA that provides tagging elements
with feature expression for a better usability. Such simplification does not reduce the expressiveness of the
language, however, it made it syntactically more consistent which helped with the reduction. Purposeful language design does wonders!

ADD FOR FINAL SUBMISSION:
-renaming in lang design!! this would have made abt in stead of ast and we would have to keep track of the bindings.

\bibliographystyle{plain}
\bibliography{bib/eric,bib/martin,bib/vdbms,bib/change,bib/vds,bib/fp,bib/error-reporting,bib/misc,bib/dblp2_short,bib/prj.bib}

\end{document}